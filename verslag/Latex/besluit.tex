%second chapter of your thesis
\chapter{Besluit}
We kunnen terugkijken op een zeer leerrijke en plezante bachelorproef. De twee aspecten van elektronica zaten in het project verwerkt, namelijk de combinatie hardware en software. We hebben ook testen uitgevoerd  om tot bepaalde inzichten te komen omtrent het plaatsen van de sensoren. We zijn gestoten op enkele problemen tijdens de uitwerking van de opdracht wat ons zeker ten goede gedaan heeft waardoor we inzichten verworven hebben in de elektronica. Sommige problemen konden we zeer snel en gemakkelijk oplossen, andere problemen hebben bloed, zweet en tranen gekost om te vinden en op te lossen. Bij het software onderdeel was het vooral de uitdaging om de PID (in ons geval PD) afregeling te optimaliseren. We waren over alle vier de circuits, die we voorgeschoteld kregen, de snelste wat het natuurlijk nog een extra prestatie maakt. Kortom, uit deze bachelorproef hebben we volgende zaken geleerd: PID afregelen, Bluetooth connectie maken met RPI, RFID-tags uitlezen, PCB ontwerpen, SMD solderen en 3D-printen. 
