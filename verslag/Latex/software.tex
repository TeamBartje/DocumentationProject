%second chapter of your thesis

\chapter{Software}

\section{Arduino}

In onze code hebben we er voor geopteerd om zo veel mogelijk gebruik te maken van zelfgemaakte bibliotheken om zo de onderdelen van ons programma zo goed mogelijk van elkaar te scheiden. Dit zorgt ervoor dat het programma overzichtelijker wordt. We gebruiken dan ook slechts \'e\'en lijn code in ons .ino bestand om een functie uit een bibliotheek aan te roepen van waaruit vervolgens het hele programma wordt aangestuurd. In Figuur \ref{fig:flowchart} vindt u de flowchart van ons programma.

\begin{figure}[h]
\centering
\includegraphics[width=0.25\textwidth]{mainFlowchart.png}
\caption{Flowchart mainprogramma}
\label{fig:flowchart}
\end{figure}


\subsection{Motor}

We zijn eerst en vooral begonnen met het schrijven van de Motor-bibliotheek. De functie bepaalPID() wordt aangeroepen vanuit de loop van het main programma genaamd voerUit.ino. We registreren, elke keer dat de loop wordt uitgevoerd, de digitale waarden aan de sensoren en plaatsen deze voortdurend in de array van integers, genaamd LFS[ ], die vervolgens eenvoudig gebruikt kan worden voor de verwerking en generatie van de effectieve correctiewaarden. We stellen dergelijke array voor als volgt: 000 001 waarbij nu enkel de meest rechtse infrarood sensor wit detecteert en alle vijf de andere sensoren zwart detecteren. We hebben het programma opgebouwd volgens het principe waarbij we aan de verschillende combinaties van sensoren die wit detecteren, verschillende foutwaarden associ\"eren zodat er in principe geen enkele rusttoestand is. Er wordt als het ware voortdurend gecorrigeerd maar in zo een klein mogelijke mate dat dit amper zichtbaar is voor ons. In ons programma krijgt de kleinste fout, namelijk de buitenste sensor die wit detecteert, de foutwaarde 1.1. In het allerslechtste geval, namelijk 000 100, waarbij een van de middelste sensoren wit detecteert, krijgt deze situatie een foutwaarde van 4.6 mee. Deze error values worden elke keer dat het programma doorlopen wordt, aangepast indien de toestand van \'e\'en van de sensoren gewijzigd werd. Deze waarden zijn niet zomaar gekozen maar we hebben uitgezocht welke combinatie van PD-constanten en error values tot de beste prestaties kon leiden. Een voorbeeld van dergelijke code vindt u in Listing \ref{listing:detectiealgoritme}. De error value die gegenereerd werd, wordt vervolgens gebruikt in de corrigeer-methode waarin de werking van ons PD-algoritme verwerkt zit.

\lstinputlisting[basicstyle=\tiny ]{detectiealgoritme.txt}
\begin{lstlisting}[caption={Het algoritme die de foutwaarden toekent aan de verschillende situaties} \label{listing:detectiealgoritme}]
\end{lstlisting}

\subsection{PD-algoritme}

Dit algoritme is vrij beknopt maar toch zeer interessant aangezien het de prestaties van de robot erg verbetert. PID is de afkorting voor Proportioneel, Integrerend en Differenti\"erend. Deze regelaar is dus in staat om verschillende soorten fouten te detecteren en in de juiste mate te reageren om zo oscillaties weg te werken en een stabiel voertuig te verkrijgen. Het proportioneel aandeel in de regelaar is een vermenigvuldiging van KP met het verschil van de waarde die je zou willen dat de sensoren meten, verminderd met de waarde die effectief geregistreerd wordt. De foutcorrectie, gegenereerd door de integrerende actie, is niet alleen afhankelijk van de grootte van de fout, maar ook van de tijd gedurende dewelke de fout zich voordoet. De I-waarde maakt het mogelijk om de offsetfout zelfs volledig te herleiden tot nul. Ook deze waarde wordt nog vermenigvuldigd met een constante namelijk KI. Als laatste hebben we nog de differenti\"erende term. Deze waarde reageert op een plotse verandering, wat de reactiesnelheid van onze robot doet toenemen. Wanneer de afwijking tussen de gewenste waarde en de gemeten waarde plots toeneemt, zal de D-waarde hevig reageren om zo een mogelijks zeer groot wordende fout te beperken in grootte. Ook in de andere richting zal de correctie afgeremd worden wanneer de gemeten waarde nadert naar de gewenste waarde. 

De P-waarde wordt gelijkgesteld aan de error value die we hierboven ge\"introduceerd hebben. De I-waarde hebben we in dit project niet gebruikt aangezien deze waarde enorm snel toeneemt en niet eenvoudig was om perfect in te stellen. Achteraf gezien bleek deze term niet noodzakelijk om een stabiel rijdende robot te verkrijgen. We maken dus gebruik van een PD-regelaar, niet van een PID-regelaar. De D-waarde wordt slechts om de 500 milliseconden vervangen door een nieuwe waarde. Mochten we deze tijdsrestrictie niet verwerkt hebben in de code, dan zou de D-waarde amper invloed hebben, aangezien ons programma honderden keren per seconde wordt uitgevoerd. Wanneer de fout gedetecteerd wordt en er geen tijdsrestrictie is, zou er onmiddellijk een nieuwe waarde worden geregistreerd die meestal opnieuw diezelfde waarde is waardoor de D-correctiewaarde wegvalt. We testten de snelheid van ons programma uit met behulp van onderstaande code om tot dergelijk inzicht te komen.
\lstinputlisting[basicstyle=\tiny ]{tijdsmeting.txt}
\begin{lstlisting}[caption={Het algoritme om de frequentie van de uitvoering van het programma te bepalen} \label{listing:tijdsmeting}]
\end{lstlisting}

De D-waarde en de P-waarde worden vervolgens nog vermenigvuldigd met de KP en KD constanten, zoals te zien in Listing \ref{listing:PID},  die we met behulp van beproeving hebben bepaald om de beste prestaties te verkrijgen. De som van deze producten wordt opgeslagen in PIDvalue en gebruikt in de rij-methode. Deze methode is verantwoordelijk voor het instellen van de snelheid per wiel, rekening houdend met de PIDvalue die we net definieerden. Dit alles wordt nog eens weergegeven in een flowchart in figuur~\ref{fig:flowchartMotor}.

\begin{figure}[h]
\centering
\includegraphics[width=0.08\textwidth]{MotorPIDFlowchart.png}
\caption{Flowchart MotorPID}
\label{fig:flowchartMotor}
\end{figure}


\lstinputlisting[basicstyle=\tiny ]{PID.txt}
\begin{lstlisting}[caption={Het PD-algoritme dat wij hanteren} \label{listing:PID}]
\end{lstlisting}



\subsection{Detectie-algoritme}

Een extra functionaliteit die we verwerkt hebben in onze code die van belang is bij de error value, is het feit dat als alle sensoren zwart detecteren dat de robot weet of hij zich nu binnen of buiten het parcours bevindt. Bij alle verschillende situaties waarin de sensoren zich kunnen bevinden, wordt er een boolean aangepast. Zo zal de boolean limitReached in alle gevallen false zijn behalve in de twee gevallen 000 100 en 001 000, dan zal de boolean true gemaakt worden. Dit zijn de situaties waarbij de robot over de buitenlijn aan het gaan is en de ene arm zich buiten het parcours bevindt en de andere arm zich binnen bevindt. Indien de fout nog een klein beetje verder toeneemt verkrijg je de situatie 000 000 waarbij de witte lijn zich tussen de sensoren bevindt. De robot zal nu zeer hevig gaan corrigeren aangezien de boolean limitReached true is. 

Wanneer we werken met twee armen ontstaan er extra problemen, namelijk naar welke arm moet er geluisterd worden en in welke richting moet de robot uitwijken wanneer alle sensoren zwart zijn. Elke keer dat er een arm wit registreert, wordt de boolean rechtersensorActief aangepast naar true of false naargelang de arm die het wit detecteert. Wanneer de sensoren volgende situatie registreren 000 001, dan wordt de boolean rechtersensorActief op true gezet. Wanneer je in dit geval los komt van de lijn, weet het programma dat het naar rechts moet gaan corrigeren om de rechterlijn terug te intercepten. Maar wanneer de boolean limitReached op true staat, wilt dat zeggen dat de situatie 000 100 zich net voordien heeft voorgedaan en het wagentje zich dus eigenlijk te ver rechts van het circuit bevindt. Nu zal het wagentje naar links moeten corrigeren met een grotere factor dan normaal aangezien de boolean limitReached deze hevige reactie genereert.


De tweede functionaliteit dat we hebben toegevoegd is het principe dat de robot weet of hij zich links of rechts van de lijn bevindt en zo dus in de juiste richting kan corrigeren. Dit was nodig om de twee armen te kunnen laten samenwerken.


\subsection {Communicatie}

In de methode bepaalPID() bevinden er zich nog twee andere methodes die aangeroepen worden, namelijk verstuurSnelheid() en verstuurTag(). Beide methodes maken gebruik van functies van de Bluetooth-bibliotheek. Bluetooth.stuurSnelheid() stuurt een waarde door die gegenereerd wordt door een methode uit de Speed-biblotheek, waarin zich het algoritme voor de registratie van de huidige snelheid van de robot bevindt die u kan raadplegen in Listing \ref{listing:snelheidsregistratie}. De snelheidswaarde wordt telkens wanneer het wieltje een rotatie maakt, aangepast met behulp van een eenvoudige snelheidsvergelijking. De snelheid wordt vervolgens \'e\'en keer per seconde met behulp van Software Serial verstuurd met de HC-05 naar de Raspberry Pi. 

\lstinputlisting[basicstyle=\tiny ]{snelheidsregistratie.txt}
\begin{lstlisting}[caption={Methode om de snelheid te registreren} \label{listing:snelheidsregistratie}]
\end{lstlisting}

De verstuurTag()-methode is een methode die instaat voor het correct versturen van de RFID-ID van de tags die zich onder de mat bevinden. We stootten daarbij op het probleem dat de mfrc522 gebruikt maakt van de SPI-pinnen. Deze pinnen waren echter reeds in gebruik door de motoraansturing die verwerkt is in onze printplaat en volledig geroute was. De RFID-lezer anders aansluiten was geen optie alsook de printplaat opnieuw routen zagen we niet onmiddellijk zitten. We hebben vervolgens geopteerd om een hulp-arduino te maken die vervolgens zou instaan voor het lezen van de RFID-tags en deze informatie vervolgens via UART, met behulp van de TX-pin, te verzenden naar de RX-pin van de hoofd-Arduino. Hierdoor hebben we het probleem van de SPI-pinnen kunnen oplossen en hebben we via een eenvoudig protocol, communicatie tot stand gebracht tussen beide Arduino's. De ID-waarde die de hoofd-Arduino ontvangt, wordt vervolgens via Bluetooth verzonden naar de Raspberry Pi.

\lstinputlisting[basicstyle=\tiny ]{communicatie.txt}
\begin{lstlisting}[caption={Versturen van data via de HC-05} \label{listing:communicatie}]
\end{lstlisting}

\section{Raspberry Pi}


Om de Raspberry Pi werkende te krijgen, hebben we eerst enkele packages moeten installeren om een seri\"ele Bluetooth communicatie mogelijk te maken. Eerst en vooral installeerden we een Bluetooth Interface met behulp van onderstaand commando.

\lstinputlisting[ ]{installbluetooth.txt}
\begin{lstlisting}[ ]
\end{lstlisting}

We kunnen nu via preferences onze Bluetooth apparaten beheren en pairen met de HC-05 van onze Arduino. Van zodra we verbonden zijn, komt er een melding dat we gebruik maken van bijvoorbeeld rfcomm0. Deze naam hebben we ook nodig om ons Python programma te laten samenwerken met de Bluetooth-chip op de Raspberry Pi. 

Het Python programma is bij ons vrij beknopt. Het is een programma dat pollt of er data binnenkomt via Bluetooth en indien er data binnenkomt, deze vervolgens onmiddellijk afprint. De Raspberry Pi ontvangt slechts op twee momenten data, namelijk de snelheid die elke seconde doorgestuurd wordt en tijdens het overrijden van een RFID-tag. Het Python programma waarvan wij gebruik gemaakt hebben vindt u in Listing \ref{listing:raspberrypicode}

\lstinputlisting[basicstyle=\tiny ]{raspberrypicode.txt}

\begin{lstlisting}[caption={Code die wordt uitgevoerd door de Raspberry Pi} \label{listing:raspberrypicode}]
\end{lstlisting}



